\documentclass[a4paper,11pt]{letter}
\usepackage{geometry}
\geometry{margin=1in}
\usepackage{setspace}
\usepackage{indentfirst}
\onehalfspacing

\begin{document}
	
	
	\begin{letter}{\Large WNBA Team Owner \& General Manager}
		
		\opening{Esteemed WNBA Team Owner and General Manager,}
		
		We are writing to present a data-driven strategic framework designed to enhance your franchise’s on-court competitiveness while ensuring long-term financial sustainability—one that is rooted in our dynamic optimization model, which has been rigorously developed and validated using real WNBA data and thousands of simulated operational scenarios. Our model is built to support your team’s decisions across multiple seasons, adapting to changes in the league environment, team performance, and financial constraints to maintain a balance between winning and profitability over time.
		
		Central to our recommendation is a balanced roster construction strategy, which our model identifies as the foundation of both competitive success and financial health. We advise allocating resources to retain 2–3 elite players, whose performance directly drives team success and fan engagement, while complementing this core group with 9–10 reliable, cost-efficient role players to meet the league-mandated roster size of 11–12 players per season. This structure distributes the salary cap such that 40\% is allocated to elite talent to keep your team competitive, and 60\% to supporting players to control expenditure—our model tests confirm this allocation yields an optimal balance: approximately 22 regular-season wins annually, paired with a projected annual profit of \$1.28 million, while avoiding the financial strain of overinvesting in star talent or the competitive decline of underinvesting.
		
		For ticket revenue optimization, our model recommends a dynamic pricing strategy tailored to both market demand and your team’s on-court momentum: an initial average price of \$35 for early-season games to build fan attendance and loyalty, with a gradual increase to \$48 for later-season contests if the team maintains competitive momentum. This approach, validated by our model’s analysis of historical attendance and pricing data, balances short-term revenue generation with long-term fan retention, and we project it can increase total ticket revenue by 15\% relative to a fixed-pricing model.
		
		It is important to acknowledge the inherent trade-offs in this strategy, which our model has clearly identified: reallocating additional cap space to elite players to pursue higher win totals (e.g., increasing wins from 18 to 25) would result in a 32\% reduction in annual profit, as the cost of additional star talent outweighs the revenue gains from more wins. Conversely, reducing investment in star talent to prioritize cost savings would decrease win totals by approximately 7\% over time, eroding fan engagement and long-term revenue potential. The primary operational risk is over-reliance on key players—our model simulates that a 20\% increase in injury probability for star players would reduce win totals by 4.2\% and profit by 2.8\%, though our recommended roster depth is specifically designed to mitigate this impact to manageable levels.
		
		The strength of our framework lies in its adaptability, a key feature of our dynamic optimization model: in anticipation of league expansion (projected to 18 teams by 2030), the model automatically adjusts talent acquisition strategies—including a 30\% increase in young player signings—to maintain competitiveness without exceeding financial constraints. Additionally, our model fully accounts for all WNBA regulatory requirements, including salary cap limitations and roster size mandates, ensuring that every recommendation is not only strategically sound but also practical and compliant, ready for implementation.
		
		In summary, this strategy, grounded in our rigorously validated dynamic model, positions your franchise to remain playoff-eligible while avoiding unsustainable financial practices. It delivers a balanced approach that supports both short-term competitive goals and long-term organizational stability, aligning every decision with the dual priorities of winning and profitability.
		
		We welcome the opportunity to discuss the details of this framework further, share additional insights from our model, and refine recommendations based on your strategic priorities for the franchise.
		
		\closing{Sincerely, \\ Members of Team 2600811}
		
	\end{letter}
	
\end{document}